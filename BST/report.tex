
\documentclass[UTF8]{ctexart}
\usepackage{geometry, CJKutf8}
\geometry{margin=1.5cm, vmargin={0pt,1cm}}
\setlength{\topmargin}{-1cm}
\setlength{\paperheight}{29.7cm}
\setlength{\textheight}{25.3cm}


% useful packages.
\usepackage{amsfonts}
\usepackage{amsmath}
\usepackage{amssymb}
\usepackage{amsthm}
\usepackage{enumerate}
\usepackage{graphicx}
\usepackage{multicol}
\usepackage{fancyhdr}
\usepackage{layout}
\usepackage{listings}
\usepackage{float, caption}

\lstset{
    basicstyle=\ttfamily, basewidth=0.5em
}

% some common command
\newcommand{\dif}{\mathrm{d}}
\newcommand{\avg}[1]{\left\langle #1 \right\rangle}
\newcommand{\difFrac}[2]{\frac{\dif #1}{\dif #2}}
\newcommand{\pdfFrac}[2]{\frac{\partial #1}{\partial #2}}
\newcommand{\OFL}{\mathrm{OFL}}
\newcommand{\UFL}{\mathrm{UFL}}
\newcommand{\fl}{\mathrm{fl}}
\newcommand{\op}{\odot}
\newcommand{\Eabs}{E_{\mathrm{abs}}}
\newcommand{\Erel}{E_{\mathrm{rel}}}

\begin{document}

\pagestyle{fancy}
\fancyhead{}
\lhead{付逸飞,3230105655}
\chead{数据结构与算法第四次作业}
\rhead{Nov.4th, 2024}

\section{remove函数的实现和测试思路}

\begin{itemize}
\item remove设计思路
\item 首先我实现了detachMin函数,
\item 里面先递归地找到最小节点的位置
\item 再判断它右边是否还有节点,如果有就替换为右边的节点,没有就替换为左边的nullptr
\item 同时oldNode用于删除节点,Node用于保存数据,然后返回这个数据
\item 然后detachMin函数实现后,remove中有两个节点的节点删除只需要替换其中的元素即可

\end{itemize}

\section{测试的结果}
我插入了一个数,打印了一个为1、2、6、7、9、12、13、20的树,根节点为9

删除1之后,打印为2、6、7、9、12、13、20

再删除6之后,打印为2、7、9、12、13、20

之后删除12后,显示为2、7、9、13、20

接着先后插入了12,11两个节点

再删除根节点9后、打印为2、7、11、12、13、20

最后删除节点12,打印为2、7、11、13、20

测试结果一切正常,均为前序遍历的结果。

我用 valgrind 进行测试,没有发现内存泄露。


\end{document}

%%% Local Variables: 
%%% mode: latex
%%% TeX-master: t
%%% End: 
