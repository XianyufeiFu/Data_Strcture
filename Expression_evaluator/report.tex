\documentclass[UTF8]{ctexart}
\usepackage{geometry, CJKutf8}
\geometry{margin=1.5cm, vmargin={0pt,1cm}}
\setlength{\topmargin}{-1cm}
\setlength{\paperheight}{29.7cm}
\setlength{\textheight}{25.3cm}


% useful packages.
\usepackage{amsfonts}
\usepackage{amsmath}
\usepackage{amssymb}
\usepackage{amsthm}
\usepackage{enumerate}
\usepackage{graphicx}
\usepackage{multicol}
\usepackage{fancyhdr}
\usepackage{layout}
\usepackage{listings}
\usepackage{float, caption}

\lstset{
    basicstyle=\ttfamily, basewidth=0.5em
}

% some common command
\newcommand{\dif}{\mathrm{d}}
\newcommand{\avg}[1]{\left\langle #1 \right\rangle}
\newcommand{\difFrac}[2]{\frac{\dif #1}{\dif #2}}
\newcommand{\pdfFrac}[2]{\frac{\partial #1}{\partial #2}}
\newcommand{\OFL}{\mathrm{OFL}}
\newcommand{\UFL}{\mathrm{UFL}}
\newcommand{\fl}{\mathrm{fl}}
\newcommand{\op}{\odot}
\newcommand{\Eabs}{E_{\mathrm{abs}}}
\newcommand{\Erel}{E_{\mathrm{rel}}}

\begin{document}

\pagestyle{fancy}
\fancyhead{}
\lhead{付逸飞, 3230105655}
\chead{数据结构与算法项目作业}
\rhead{Dec.7th, 2024}


\section{四则混合运算器的设计思路}

整体设计程序时,我将程序拆分为大致三个板块————输入,中缀表达式转化(\texttt{rechange\_original}函数),计算。
其中输入部分比较麻烦,本次项目作业的难度也集中在这个部分,需要将一个读进来的字符串转化为一个单独的字符和数,同时如何同时储存数和字符也遇到了困难。
为了方便后续处理和调试,我没有用STL里面的stack,而是自己写了一个结构体数组来模拟stack,因为这样既可以从前往后一顺遍历同时又可以模拟stack,比较方便。
至于为什么用结构体,因为一开始我忘了指针也能开数组hhh,所以就没用指针,然后结构体里边只有单独的三个变量,用起来也比较方便可以减少出现一些奇奇怪怪的错误。
结构体里面的三个变量分别为priority、ch(存操作符)和value(存数字数值)。数字的优先级我设为最低为-1,只是方便了识别。\par

输入函数\texttt{read\_original}的思路主要是遇到不同的情况分别讨论,遇到数字则用一个v来记录以读入数字的数字,然后遇到操作符则将累计读入的数压入stack中。
遇到操作符则直接读入,然后就是一些读入负数、小数和科学计数法的处理,主要是用了几个flag判断是哪种情况,分别处理。
同时也识别一些显然的非法输入,比如连续两个e或者小数点这些。\par

后面两个函数设计思路则与老师上课的时候一样了。\texttt{rechange\_original}函数遇到数字则将数字压入rechange\_stack中,遇到操作符则压入单独的op\_stack中。
当要压入的操作符比栈顶的元素优先级低时,不断弹出元素进rechange\_stack中知道比栈顶元素优先级低或相同再压入,左括号优先级为0在操作符中最低,只有右括号能与其配对。
原读入的序列结束时,将剩余的栈中操作符一次pop进rechange\_stack中。顺便检查有无剩余括号,如有则立马退出并标记。\par

最后的是计算这个后缀表达式了(\texttt{calculate}函数)这个比较简单,将原来的rechange\_stack从前往后遍历,遇到数字直接压入calculate\_stack中,
遇到操作符pop出两个数进行运算,不够则判断为非法式子并退出,最后若剩余超过一个数也判断非法,这样基本能检测出表达式的正误了。

\section{相关说明}

程序将来连续两个-号视为合法,前一个为-后一个表示负数,其他如++、**、-+均视为非法。\par
程序科学计数法e后只能为整数,能识别负数但不能在e后加其他操作符。\par
程序将式子中含有除e、.、数字和操作符(+、-、*、/)以外的字符均视为非法。\par
小数点.前必须有数字,否则视为非法。

\end{document}

%%% Local Variables: 
%%% mode: latex
%%% TeX-master: t
%%% End: 
