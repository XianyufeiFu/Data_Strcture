\documentclass[UTF8]{ctexart}
\usepackage{geometry, CJKutf8}
\geometry{margin=1.5cm, vmargin={0pt,1cm}}
\setlength{\topmargin}{-1cm}
\setlength{\paperheight}{29.7cm}
\setlength{\textheight}{25.3cm}


% useful packages.
\usepackage{amsfonts}
\usepackage{amsmath}
\usepackage{amssymb}
\usepackage{amsthm}
\usepackage{enumerate}
\usepackage{graphicx}
\usepackage{multicol}
\usepackage{fancyhdr}
\usepackage{layout}
\usepackage{listings}
\usepackage{float, caption}

\lstset{
    basicstyle=\ttfamily, basewidth=0.5em
}

% some common command
\newcommand{\dif}{\mathrm{d}}
\newcommand{\avg}[1]{\left\langle #1 \right\rangle}
\newcommand{\difFrac}[2]{\frac{\dif #1}{\dif #2}}
\newcommand{\pdfFrac}[2]{\frac{\partial #1}{\partial #2}}
\newcommand{\OFL}{\mathrm{OFL}}
\newcommand{\UFL}{\mathrm{UFL}}
\newcommand{\fl}{\mathrm{fl}}
\newcommand{\op}{\odot}
\newcommand{\Eabs}{E_{\mathrm{abs}}}
\newcommand{\Erel}{E_{\mathrm{rel}}}

\begin{document}

\pagestyle{fancy}
\fancyhead{}
\lhead{付逸飞, 3230105655}
\chead{数据结构与算法第七次作业}
\rhead{Dec.2nd, 2024}



\begin{table}[h!]
\centering
\begin{tabular}{|c|c|c|}
\hline
 & my heapsort time & std::sort\_heap time \\
\hline
random sequence & 220ms & 222ms \\
\hline
ordered sequence & 111ms & 84ms \\
\hline
reverse sequence & 111ms & 103ms \\
\hline
repetitive sequence & 160ms & 160ms \\
\hline
\end{tabular}
\caption{堆排序与 std::sort_heap 的时间对比}
\label{tab:example}
\end{table}

\section{对HeapSort的阐述和分析}
在\texttt{heapSort} 类的\texttt{sort}函数中,我使用将向量转换为堆,然后使用 \texttt{std::pop\_heap} 逐个将堆顶元素(最大元素)移到当前范围的末尾,反复调用N次,每次缩短一个范围。
最终,令向量 v1 中的元素按升序排列。
关于check函数:我希望check函数实现检查向量是否按升序排列的功能,所以我使用了一个for循环,检查v1[i]是否大于v1[i+1],如果是则立即返回false,检查完成后返回true。  
然后我设计了四个函数来生成随机、有序、逆序、部分重复四种序列。
对于测试流程,我首先生成一个随机序列,然后调用\texttt{sort}函数对其进行排序,然后调用check函数检查排序结果是否正确。同时记录排序时间和用 \texttt{std::pop\_heap} 进行排序的时间并比较。
同理操作有序、逆序、部分重复四种序列。最后将结果输出到文件中。
效率表格如下


\section{补充}

经过理论计算,\texttt{sort}的时间复杂度应该是$O(n\log n)$,而\texttt{std::sort\_heap}的时间复杂度也是$O(n\log n)$。
理论上两者的时间复杂度是相同的。
但是在实际测试中,我发现\texttt{std::sort\_heap}的时间一般要比\texttt{heapSort}的时间要短,
猜测可能是我调用的是STL的封装的函数然后进行了二次封装,会导致每次多了几个$O(1)$的动作,从而导致总的系数会变大一点,没有一次性封装的\texttt{std::pop\_heap}高效。
同时,STL库内部的优化也会使得其更快一点。



\end{document}

%%% Local Variables: 
%%% mode: latex
%%% TeX-master: t
%%% End: 
